\lhead{\emph{\leftmark}}  
\chapter{Related Work}
\label{chap:related}
This chapter surveys previous work in Reverse engineering approaches generating UML. A common theme in much of this work is a choice between two approaches: static and dynamic analysis. These concepts have been presented in Chapter 2. The following section describes the literature review methodology. We then present the results of our findings and a comparison between the different approaches and our own approach.

\section{Literature Review Methodology}

This study has been undertaken as a systematic literature review based on the guidelines proposed by Kitchenham \cite{kitchenham2004procedures}. Key parts of this systematic literature review are presented in this thesis.

\subsection{Research Questions}

The main goal of this systematic review was to identify and classify different techniques for reverse engineering to UML. Specifically, we target the reverse engineering to UML of software systems by means of model transformations. The high-level research question addressed by this study is:

\textbf{RQ1.} What model transformation techniques and/or methodologies for reverse engineering to UML can be identified from the literature?

\subsection{Search process}

To search the databases the, a set of strings was created for each of the research questions based on keywords extracted from the research questions and augmented with synonyms.
We designed a two-phase systematic review. In both phases, we first selected the related work using the search engines and cited references in the Table 1. Afterwards, we performed an analysis on the related work. In the second phase, we also conduct-ed a detailed review of a selected subset of initial results. To assure there is not already a literature review answering our research questions, in the first phase, we looked at existing surveys and literature review papers. In the second phase, we focused on studying the existing work on reverse engineering to UML. 

The sources for the search were chosen such that they included journals and conferences focusing on software engineering and program comprehension. 

The search resulted in an extensive list of potential papers. To ensure that all papers included in the review were related to the research questions, we defined detailed inclusion and exclusion criteria.  

\subsection{First Phase Queries}


\subsection{Second Phase Queries}


\subsection{Inclusion and exclusion criteria}







